%--- Project Euler 317 formula derivation.
\documentclass[a4paper,12pt]{article}

% Packages
\usepackage[utf8]{inputenc}

% Font & Document Look
\usepackage{wrapfig}
\usepackage[margin=0.5in]{geometry}
\usepackage{url}
\usepackage{graphicx}
\usepackage{xcolor}
\usepackage[parfill]{parskip}
\usepackage{dirtytalk}
\usepackage{csquotes}
\usepackage{multirow}
\usepackage{multicol}
\usepackage{booktabs}
\usepackage{tabularx}
\usepackage[]{microtype}
\usepackage{float}
\usepackage{listings}
\usepackage{verbatim}
\usepackage{siunitx}
\usepackage{amsmath}
\usepackage{amssymb}
\usepackage{gensymb}
% References
\usepackage[round,comma, authoryear]{natbib}
\usepackage{har2nat}
\bibliographystyle{agsm}

\newcommand{\ivec}{\mathbf{\hat{i}}}
\newcommand{\jvec}{\mathbf{\hat{j}}}
\begin{document}
This document is a derivation of the formula used in Euler317.

We begin with a vector description of projectile motion:
   
\[\vec{r} = u t\cos\theta \ivec + ut\sin\theta + \frac{1}{2}at^2 + h \jvec\]
where \(u, \theta, h, t\) are initial velocity, launch angle, height from which it is launched and time elapsed respectively.
We find that:
\begin{align}
    x &= u t\cos\theta \\
    y &= ut\sin\theta + \frac{1}{2}at^2 + h 
\end{align}
These can be rearranged and substituted in for \(t\).
\begin{align*}
    y &= u\left(\frac{x}{u\cos\theta}\right)\sin\theta + \frac{1}{2}a\left(\frac{x}{u\cos\theta}\right)^2 + h \\
    y &= \frac{x\sin\theta}{\cos\theta} + \frac{ax^2}{2 u^2 \cos^2\theta} + h \\
    y &= x \tan \theta - \frac{ax^2}{2u^2}\sec^2 \theta + h
\end{align*}
As acceleration is due to gravity, \(a = g\), and to make it look nice let \(u = v\).
Hence the final equation is
\[y = x \tan \theta - \frac{gx^2}{2v^2}\sec^2 \theta + h.\]
This is all well and good, but we must find an `envelope' that encompasses all parabolas for values of \(\theta\).
Wkipedia says that the envelope of the family of curves parametised by \(t\) is given as the set of points where 
\[F(t,x,y) = \frac{\delta F}{\delta t}(t,x,y) = 0.\]
We have \(F(\theta,x,y) = x \tan \theta - \frac{gx^2}{2v^2}\sec^2 \theta + h-y\), hence:
\[
    \frac{\delta F}{\delta \theta}(\theta,x,y) = x \sec^2 \theta + \frac{gx^2}{v^2}\sec^2\theta\tan \theta 
\]
by applying the chain rule. Solving \(\frac{\delta F}{\delta \theta} = 0\):
\begin{align*}
    0 &= x \sec^2 \theta + \frac{gx^2}{v^2}\sec^2\theta\tan \theta \\
    0 &= \frac{x}{v^2}\sec^2 \theta \left(v^2 - gx\tan \theta\right) \\
    \therefore \tan \theta &= \frac{v^2}{gx}
\end{align*}
Using the trig identity \(\sec^2x = 1 + \tan^2 x\), we find that \(\sec^2x = \frac{g^2x^2+v^4}{g^2x^2}\).
Substituting into \(F(\theta,x,y) = 0\):
\begin{align}
    y &= \frac{v^2}{g} -\frac{gx^2}{2v^2}\left(\frac{g^2x^2+v^4}{g^2x^2}\right) + h \nonumber\\
    y  &= \frac{v^2}{g} - \frac{g^2x^2+v^4}{2v^2g} + h\nonumber\\
    y  &= \frac{v^2}{2g} - \frac{gx^2}{2v^2} +h \label{eq:envelope}
\end{align}

To find the solid of revolution around the y-axis:
\[V = \pi\int^a_b [f(y)]^2 \space dy \]

Using equation~\ref{eq:envelope}:
\begin{align*}
    y  &= \frac{v^2}{2g} - \frac{gx^2}{2v^2} + h \\
    y - \frac{v^2}{2g} - h &= - \frac{gx^2}{2v^2} \\
    \frac{v^2}{g}(y-\frac{v^2}{2g}-h) &= x^2 \\
    x &= \sqrt{\frac{v^2}{g}(y-\frac{v^2}{2g}-h)}
\end{align*}

The bounds on the integral will be \(0\) and \(f(0)\), where \(f(x)\) is equation~\ref{eq:envelope}.

Using Julia and plugging in \(\{v,g, h\} = \{20, 9.81, 8\}\)
\end{document}